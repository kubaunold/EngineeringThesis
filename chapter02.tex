\chapter{Application}
\section{Theoretical background}

\subsection{Geometric Brownian Motion}
\todo
it's basically Wiener Process\^2

\subsection{Option}
\todo
Option can be Put or Call, it measns...

\subsection{Black-Scholes Model}
The Black--Scholes (or also called Black--Scholes--Merton) formula was created back in 1970s by 3 major economists: Fischer Black, Myron Scholes and Robert Merton (Scholes and Merton were awarded Nobel prize for economics in 1997. Certainly same would happen for Black if sadly it was not for his death in 1975).

Their model was a significant breakthrough in world of mathematical models used for pricing derivative instruments. It provides a framework for European-style option valuations, such as calls and puts.

This thesis' aim was not to goo deep into understanding the mathematical background behind the model but rather implement the model in a practical tool that one would be able to effectively use. Therefore more information and specifics can be found either in the original publication \cite{10.2307/1831029} of the model from 1973 in \textit{The Pricing of Options and Corporate Liabilities} of the Journal of Political Economy by Fischer Black and Myron Scholes.

The formula assumes that the price history of an underlying asset (in this example - a stock price) has a lognormal distribution and follows geometric Brownian motion with constant drift and volatility. 
